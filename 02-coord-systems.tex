\section{Системы координат для описания движения цели и ракеты}

Адекватное математическое описание процесса перехвата невозможно без выбора удобных систем координат (СК).
\subsection{Биконическая система координат (БСК)}
Это самая важная система для управления зенитной ракетой. Её начало находится в точке, откуда ведется управление — обычно это радиолокационная станция (РЛС), которая "видит" цель.
\begin{figure}[t]
    \centering
    \includegraphics[width=0.8\textwidth]{02-biconic.png}
    \caption{Биконическая система координат}
    \label{fig:biconic}
\end{figure}
Как видно из рисунка \ref{fig:biconic}, чтобы найти цель в пространстве, нужно знать три величины:
\begin{itemize}
    \item \textbf{Дальность} ($D_n$) — расстояние от РЛС до цели.
    \item \textbf{Азимут} ($\varphi_n$) — боковой угол, показывающий, насколько цель смещена вправо или влево от направления "прямо вперед" от РЛС.
    \item \textbf{Угол места} ($\varphi_b$) — вертикальный угол, показывающий, насколько цель выше или ниже горизонтальной плоскости.
\end{itemize}

Представьте, что вы стоите с фонариком: дальность — это как далеко объект, азимут — в какую сторону по горизонтали светить, угол места — поднимать или опускать луч фонарика. Именно так РЛС "следит" за целью.

\subsection{Связанная стартовая система координат}
Необходимые для выведения стартующей ЗУР на кинематическую траекторию метода наведения, могут осуществляться в связанной ($O_{X_{св}}$, $O_{Y_{св}}$, $O_{Z_{св}}$) и стартовой ($O_{X_{ст}}$, $O_{Y_{ст}}$, $O_{Z_{ст}}$) системах координат
\begin{itemize}
    \item \textbf{Связанная система координат} — это внутренняя система отсчёта, которая жёстко привязана к самой ракете. Её начало находится в центре масс ракеты. Оси этой системы закреплены на корпусе: ось $O_{X_{св}}$ идёт вдоль ракеты от хвоста к носу, ось $O_{Y_{св}}$ направлена вверх относительно ракеты, а ось $O_{Z_{св}}$ — вбок, чтобы все три оси были перпендикулярны друг другу. Когда ракета поворачивается в полёте, эта система координат поворачивается вместе с ней. Благодаря гироскопам, которые ориентированы по этим осям, ракета "чувствует" свои повороты и может стабилизировать своё положение.
    \begin{figure}[t]
        \centering
        \includegraphics[width=0.8\textwidth]{02-svayz.jpg}
        \caption{Связанная система координат}
        \label{fig:svayz}
    \end{figure}

    \item \textbf{Стартовая система координат} — это система, которая строится в момент пуска ракеты для наведения на цель. Её центр находится в ракете. Ось $O_{X_{ст}}$ горизонтально направлена в сторону цели, ось $O_{Y_{ст}}$ лежит в вертикальной плоскости, проходящей через ракету, а ось $O_{Z_{ст}}$ перпендикулярна этой плоскости.

    Перед стартом рассчитывают два угла: угол {$\vartheta_{cкл}$} — для поворота вправо-влево, и угол {$\gamma_{cкл}$} — для подъёма-опускания. Эти углы загружают в компьютер ракеты.
    После пуска ракета летит сама, удерживая эти углы, чтобы выйти в точку встречи с целью, где её уже точно захватит система наведения.

    \begin{figure}[t]
        \centering
        \includegraphics[width=0.8\textwidth]{02-start.jpg}
        \caption{Связанная стартовая система координат}
        \label{fig:start}
    \end{figure}

\end{itemize}

\newpage

\subsection{Скоростная система координат}
Скоростная система координат — это система, в которой положение ракеты определяется относительно направления её полёта.
Её центр находится в центре масс ракеты:
\begin{itemize}
    \item \textbf{Ось $O_{X_{V}}$} направлена туда, куда ракета летит в данный момент (по вектору скорости)
    \item \textbf{Ось $O_{Y_{V}}$} направлена вверх в вертикальной плоскости ракеты
    \item \textbf{Ось $O_{Z_{V}}$} направлена вбок, дополняя систему
\end{itemize}
Главные отличия от связанной системы:
\begin{itemize}
    \item \textbf{Связанная система} привязана к корпусу ракеты (нос всегда по оси $O_{X_{св}}$)
    \item \textbf{Скоростная система} привязана к направлению полёта (куда реально движется ракета)
\end{itemize}
Важные углы:
\begin{itemize}
    \item \textbf{Угол атаки ($\alpha$)} — угол между носом ракеты и направлением полёта (как угол между стрелой и траекторией)
    \item \textbf{Угол скольжения ($\beta$)} — угол бокового смещения направления полёта
\end{itemize}
Зачем это нужно: Именно в этой системе оценивают ошибки наведения и рассчитывают команды для рулей, чтобы ракета летела точно к цели.
\begin{figure}[t]
        \centering
        \includegraphics[width=0.8\textwidth]{02-speed.jpg}
        \caption{Скоростная система координат}
        \label{fig:speed}
    \end{figure}