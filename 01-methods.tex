\section{Основные методы наведения}

\subsection{Метод пропорционального сближения (Погони)}
Наиболее распространённый метод терминального самонаведения. Его суть заключается в том, что угловая скорость поворота вектора скорости ракеты \(\dot{\theta}\) прямо пропорциональна угловой скорости вращения \textbf{линии визирования} \(\dot{q}\) -- воображаемой прямой, соединяющей ракету (точка \(B\)) и цель (точка \(A\)).

\begin{equation} \label{eq:pursuit}
	\dot{\theta} = k \cdot \dot{q},
\end{equation}
\noindent где:
\begin{itemize}
	\item \(\dot{\theta}\) -- угловая скорость вектора скорости ракеты \(\vec{V_p}\) (требуемое управляющее воздействие),
	\item \(\dot{q}\) -- угловая скорость линии визирования (измеряется головкой самонаведения -- ГСН),
	\item \(k > 1\) -- коэффициент пропорциональности (\textbf{коэффициент упреждения}). Чем больше \(k\), тем прямее будет траектория и больше упреждение.
\end{itemize}

Из уравнения (\ref{eq:pursuit}) выводится формула для требуемого \textbf{нормального (поперечного) ускорения} \(a_n^{req}\) ракеты, которое обеспечивает данный поворот вектора скорости:
\begin{equation} \label{eq:pursuit_accel}
	a_n^{req} = V_p \cdot \dot{\theta} = k \cdot V_p \cdot \dot{q},
\end{equation}
\noindent где \(V_p\) -- модуль скорости ракеты (\(V_p = |\vec{V_p}|\)).

\begin{figure}[H]
	\centering
	\includegraphics[width=0.5\linewidth]{01-proporch.png}
	\caption{Кинематика метода пропорционального сближения. Вектор скорости ракеты \(\vec{V_p}\) поворачивается с угловой скоростью \(\dot{\theta}\), пропорциональной угловой скорости линии визирования \(\dot{q}\).}
	\label{fig:pursuit}
\end{figure}

\textbf{Преимущества:} Высокая точность против маневрирующих целей.
\textbf{Недостатки:} При \(k \to 1\) (чистая погоня) траектория сильно изогнута, что ведёт к большим энергозатратам и возможному промаху при высокой скорости сближения.

\subsection{Метод параллельного сближения}
Данный метод является идеальным предельным случаем метода пропорционального сближения. Его цель — полностью устранить вращение линии визирования. Ракета (точка \(B\)) должна двигаться так, чтобы линия визирования на цель (точка \(A\)) оставалась параллельной самой себе, т.е. её угловая скорость была равна нулю. Фактически, ракета летит не на текущее положение цели, а в расчётную точку встречи (точка \(M\)).

\begin{equation} \label{eq:parallel}
	\dot{q} = 0
\end{equation}

\begin{figure}[H]
	\centering
	\includegraphics[width=0.5\linewidth]{01-paralel.png}
	\caption{Кинематика метода параллельного сближения. Линия визирования не вращается (\(\dot{q} = 0\)), вектор скорости ракеты \(\vec{V_p}\) направлен так, чтобы ракета (точка \(B\)) двигалась прямо в точку упреждённой встречи с целью (точка \(M\)).}
	\label{fig:parallel}
\end{figure}

На практике абсолютный нуль \(\dot{q}\) недостижим, поэтому используется \textbf{модифицированный метод параллельного сближения}, где система управления стремится поддерживать минимально возможное, близкое к нулю значение \(\dot{q}\).

\textbf{Преимущества:} Оптимальная (прямая или близкая к прямой) траектория, минимальные энергозатраты и время перехвата.
\textbf{Недостатки:} Крайне высокие требования к точности начального целеуказания, расчёта точки упреждения и быстродействию системы управления.

\subsection{Трёхточечный метод (метод совмещения)}
Классический метод для телеуправления первого рода. Траектория ракеты жёстко привязана к наземному пункту управления (точка встречи \(P_0\)). Ракета должна постоянно находиться на прямой линии, соединяющей пункт управления (точка встречи) и цель (точка \(Ц_0\)).

\begin{equation} \label{eq:3point}
	\theta_m(t) = \theta_t(t),
\end{equation}
\noindent где \(\theta_m\) и \(\theta_t\) -- угловые положения ракеты и цели относительно пункта управления.

Управляющая команда формируется как сигнал рассогласования между текущим \(\theta_m(t)\) и требуемым \(\theta_t(t)\) угловым положением ракеты в системе координат пункта управления.

\textbf{Преимущества:} Простота технической реализации, отсутствие необходимости в сложной ГСН на борту.
\textbf{Недостатки:} Сильно искривлённая траектория (особенно на встречных курсах), большой расход скорости ракеты на манёвр, низкая эффективность против резко маневрирующих целей.

\subsection{Метод половинного спрямления}
Компромиссный метод, применяемый в системах телеуправления. Требует, чтобы ракета находилась на линии, которая делит пополам угол между текущей и будущей линией визирования «Пункт управления (точка встречи) — Цель (\(Ц_0\))».

\begin{equation} \label{eq:half}
	\theta_m(t) = \theta_t(t) - \frac{1}{2}[\theta_t(t) - \theta_{t}^{future}(t)],
\end{equation}
 или в дифференциальной форме: угловая скорость ракеты относительно пункта управления должна быть вдвое меньше угловой скорости цели:
\begin{equation}
	\dot{\theta}_m = 0.5 \cdot \dot{\theta}_t.
\end{equation}

Траектория ракеты получается более пологой, чем в трёхточечном методе, что снижает требуемую перегрузку и увеличивает дальность действия комплекса.
