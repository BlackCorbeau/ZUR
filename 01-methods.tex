\section{Задача стрельбы зенитными управляемыми ракетами}

\textbf{Определение:} Задача стрельбы ЗУР --- \textbf{поражение воздушной цели с требуемой эффективностью} в условиях противодействия, маневра цели и ограничений самого ракетного комплекса.

\subsection{Основные составляющие задачи}

\subsubsection{Определение ошибок наведения и параметров закона поражения цели}
\begin{itemize}
	\item Анализ факторов, вызывающих отклонение ракеты от требуемой траектории.
	\item Расчёт зоны поражения с учётом характеристик боевой части и взрывателя.
\end{itemize}

\subsubsection{Расчёт показателей эффективности стрельбы}
\begin{itemize}
	\item Определение \textbf{вероятности поражения цели} одной или несколькими ракетами.
	\item Расчёт \textbf{математического ожидания числа сбитых самолётов} в групповой цели.
\end{itemize}

\subsubsection{Оценка пространственных и временных возможностей ЗРК}
\begin{itemize}
	\item Определение \textbf{зоны поражения} по дальности, высоте и параметру.
	\item Расчёт \textbf{времени реакции} комплекса, циклограммы работы.
	\item Оценка возможности обстрела нескольких целей.
\end{itemize}

\subsubsection{Подготовка стрельбы и ведение огня}
\begin{itemize}
	\item \textbf{Организация стрельбы:} Разведка целей, целераспределение, выбор метода наведения.
	\item \textbf{Ведение огня:} Учёт маневра целей (вираж, пикирование, горка, разгон/торможение).
	\item \textbf{Противодействие:} Помехам и тактическим приёмам противника (совмещение отметок на индикаторе, резкое изменение скорости).
\end{itemize}

\newpage

\section{Типы систем управления зенитными управляемыми ракетами (ЗУР)}

\subsection{Телеуправление (командное)}
\begin{itemize}
	\item \textbf{Принцип работы:} Управление осуществляется \textbf{с наземного пункта наведения}.
	\item \textbf{Техническая реализация:}
	\begin{itemize}
		\item \textbf{Станция наведения (СНР)} отслеживает цель и ракету.
		\item Формирует команды, передаваемые на борт ракеты.
		\item Используется \textbf{двухсторонняя радиолиния} или \textbf{лазерный луч}.
	\end{itemize}
	\item \textbf{Недостаток:} Зависимость от канала связи, помехозащищенность ограничена.
\end{itemize}
\begin{figure}[h!]
	\centering
	\begin{minipage}{0.45\textwidth}
		\centering
		\includegraphics[width=\textwidth]{01-1.png}
		\caption{Командная система телеуправления 1-го рода (лазерный луч)}
		\label{fig:sys_composition}
	\end{minipage}
	\hfill
	\begin{minipage}{0.45\textwidth}
		\centering
		\includegraphics[width=\textwidth]{01-2.png}
		\caption{Командная система телеуправления 2-го рода (двухсторонняя радиолиния)}
		\label{fig:guidance_scheme}
	\end{minipage}
\end{figure}

\subsection{Самонаведение}
\begin{itemize}
	\item \textbf{Принцип работы:} Ракета \textbf{самостоятельно} наводится на цель с помощью бортовой аппаратуры.
	\item \textbf{Типы головок самонаведения (ГСН):}
	\begin{itemize}
		\item \textbf{Пассивное:} Ракета принимает излучение цели (тепловое, радиолокационное).
		\item \textbf{Активное:} Ракета имеет собственный излучатель и приёмник.
		\item \textbf{Полуактивное:} Цель подсвечивается наземной РЛС, ракета принимает отражённый сигнал.
	\end{itemize}
	\item \textbf{Достоинство:} Автономность на конечном участке.
	\item \textbf{Недостаток:} Зависимость от метеоусловий, возможность постановки помех.
\end{itemize}

\subsection{Комбинированное управление}
\begin{itemize}
	\item \textbf{Принцип работы:} Сочетание \textbf{различных систем} на разных этапах полёта.
	\item \textbf{Примеры:}
	\begin{itemize}
		\item Телеуправление на первом участке + самонаведение на конечном.
		\item Телеуправление первой и второй ступенями.
	\end{itemize}
	\item \textbf{Решаемые задачи:}
	\begin{itemize}
		\item \textbf{Сопряжение траекторий} при переходе с одного способа на другой.
		\item \textbf{Обеспечение захвата цели} головкой самонаведения.
		\item Использование \textbf{одной бортовой аппаратуры} на разных этапах.
	\end{itemize}
\end{itemize}

\subsection{Автономные системы управления}
\begin{itemize}
	\item \textbf{Принцип работы:} Управляющий сигнал формируется \textbf{на борту ракеты} по заранее заданной программе.
	\item \textbf{Особенность:} Не требует информации от цели или пункта управления после старта.
	\item \textbf{Применение:} На \textbf{начальном участке траектории} для вывода ракеты в заданную область пространства.
\end{itemize}