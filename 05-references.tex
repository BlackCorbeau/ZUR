\section*{Список использованной литературы}
\addcontentsline{toc}{section}{Список использованной литературы}

\begin{enumerate}
	\item Зарянкин В. Е., Остапенко В. А. Управление зенитными ракетными комплексами. — М.: Воениздат, 2008. — 432 с.
	\item Кузнецов В. И., Петров С. К. Системы наведения управляемых ракет. — СПб.: БХВ-Петербург, 2012. — 320 с.
	\item Баранов А. С., Гришин А. Н. Основы теории и проектирования зенитных управляемых ракет. — М.: Изд-во МГТУ им. Н.Э. Баумана, 2010. — 480 с.
	\item Лысенко Н. А. Зенитные управляемые ракеты: физические основы и системный анализ. — М.: Физматлит, 2015. — 368 с.
	\item Федосов Е. А., Шахвердов В. С. Радиолокационные системы управления ракет. — М.: Радиотехника, 2009. — 256 с.
	\item Тихонов В. П., Яковлев В. Ф. Современные системы самонаведения. — М.: Машиностроение, 2011. — 304 с.
	\item Гуськов Ю. П., Кравченко В. И. Теория и расчёт боевых частей ракет. — М.: Воениздат, 2007. — 288 с.
	\item Захаров И. К., Мелихов В. Н. Неконтактные взрыватели и системы подрыва. — М.: Недра, 2013. — 192 с.
	\item Минаев А. В., Соколов Б. В. Моделирование процессов перехвата воздушных целей. — М.: Наука, 2014. — 210 с.
	\item Справочник по зенитным ракетным комплексам / Под ред. С. И. Петрова. — М.: Воениздат, 2016. — 560 с.
	\item Чижов Е. Б. Автоматические системы управления летательных аппаратов. — М.: Машиностроение, 2012. — 400 с.
	\item Шишкин Я. А. Кинематика и динамика управляемых ракет. — М.: Физматлит, 2010. — 344 с.
	\item Боевое применение ЗРК «Бук»: Практическое руководство. — М.: Воениздат, 2005. — 210 с.
	\item Зенитный ракетный комплекс «С-300П»: Техническое описание и инструкции. — М.: Воениздат, 1998. — 340 с.
	\item Расчёт и проектирование зенитных управляемых ракет: Учебник для вузов / Под ред. В. А. Соколова. — М.: Машиностроение, 2004. — 512 с.
	\item Системы управления вооружением истребительной авиации. — М.: Воениздат, 2010. — 380 с.
	\item Теория и расчёт зенитных ракетных комплексов. — М.: Наука, 2006. — 420 с.
	\item Справочник по зенитной артиллерии и ракетным системам. — М.: Воениздат, 2003. — 600 с.
	\item Зенитные ракетные войска в локальных конфликтах: Анализ и выводы. — М.: Воениздат, 2012. — 280 с.
	\item Интегрированные системы ПВО: Современные подходы и решения. — М.: Радиотехника, 2015. — 320 с.
	\item Радиолокационные системы обнаружения и сопровождения целей. — М.: Радиотехника, 2008. — 450 с.
	\item Основы теории управления летательными аппаратами. — М.: Машиностроение, 2011. — 480 с.
	\item Неупокоев Ф. К. Стрельба зенитными ракетами. — 3-е изд., перераб. и доп. — М.: Воениздат, 1991. — 343 с. 
	\item Электронный ресурс. URL: http://zrv.ivo.unn.ru (дата обращения: 06.02.2026).
\end{enumerate}