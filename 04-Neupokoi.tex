\section{Физические основы поражения цели зенитной управляемой ракетой}
Заключительным и решающим этапом работы зенитной управляемой ракеты является поражение цели её боевой частью (БЧ). Эффективность этого этапа определяется типом БЧ, её конструкцией и соответствием характеристикам воздушной цели.
По способу воздействия боевые части ЗУР подразделяются на осколочные, фугасные и кумулятивные, причём наибольшее применение нашли осколочные БЧ. Важнейшим конструктивным отличием является деление осколочных БЧ на ненаправленного и направленного действия.
\begin{itemize}
    \item \textbf{Боевые части ненаправленного действия} поражают цель равномерно во всех направлениях. Их преимущество — простота применения: для подрыва достаточно информации о моменте наибольшего сближения с целью.
    \item \textbf{Боевые части направленного действия} обеспечивают большее поражающее действие в заданном секторе пространства. За счёт концентрации потока осколков они в несколько раз увеличивают плотность поражающих элементов в зоне нахождения цели при той же массе БЧ, что делает их энергетически более выгодными. Однако их применение требует дополнительной информации для правильной ориентации в пространстве в момент подрыва.
\end{itemize}

\begin{figure}[h!]
    \centering
    \includegraphics[width=0.8\textwidth]{04-oskolki.jpg}
    \caption{Зоны разлёта осколков боевой части направленного действия}
    \label{fig:oskolki}
\end{figure}

Радиус зоны поражения зависит от скорости поражающих элементов, их массы и формы. Для поражения цели осколок должен обладать определённой кинетической энергией:
\vspace{1cm}
$\displaystyle \frac{(m_{\text{оск}}*V_{\text{оск}}^2)}{2} \ge {\text{Э}}_{\text{y}}*{\text{s}}*{\text{h}}$, где h - толщина прешрады, s - площадь пробоины, ${\text{Э}_\text{y}}$ - удельная энергия вытеснения единицы объёма материала преграды

Однако для реализации поражающего потенциала БЧ критически важен точный момент её подрыва. В современных ЗУР это обеспечивается неконтактными взрывателями (НВ), автономно определяющими оптимальную точку детонации. Наиболее распространены радиовзрыватели, которые по принципу действия делятся на:
\begin{itemize}
    \item \textbf{Активные} (ракета сама облучает цель радиосигналом и принимает отражение);
    \item \textbf{Полуактивные} (принимают сигнал, отражённый от цели, которую облучает внешний источник, например, наземная РЛС);
    \item \textbf{Пассивные} (реагируют на собственное радиоизлучение цели, эффективны против постановщиков помех);
\end{itemize}
Ключевым требованием является согласование области срабатывания взрывателя с областью возможного поражения БЧ. Область срабатывания — это пространство вокруг ракеты, при попадании в которое центра цели взрыватель инициирует подрыв. Если эти области не совпадают, подрыв происходит либо преждевременно, либо с опозданием, что резко снижает вероятность поражения даже при идеально рассчитанной БЧ.
Конечной же мерой эффективности всего процесса перехвата служит уязвимость конкретной воздушной цели — её способность противостоять поражающим факторам. Уязвимость не является константой; она зависит от:

\begin{itemize}
    \item \textbf{прочности конструкции и наличия средств защиты} (броня, протекторы баков);
    \item \textbf{расположения и дублирования жизненно важных элементов} (двигатели, топливные системы, экипаж);
    \item \textbf{внешних условий встречи:} высоты полёта и взаимной ориентации цели и ракеты;
\end{itemize}
Таким образом, физические основы поражения цели ЗУР представляют собой последовательную цепь взаимосвязанных условий: создание направленного поля высокой плотности энергии (конструкция БЧ) → его точная активация в нужной точке пространства (работа взрывателя) → преодоление этим полем конструкционной устойчивости объекта (учёт уязвимости цели). Нарушение любого из этих звеньев сводит на нет эффективность всей системы.
