\section{Введение}

Перехват воздушных целей зенитными управляемыми ракетами (ЗУР) является одной из ключевых задач современной противовоздушной обороны. Эффективность такого перехвата зависит от множества факторов: точности наведения, скорости и маневренности ракеты, правильности выбора системы координат, типа системы управления и физических принципов поражения цели.

Целью данного реферата является систематизированное изложение физических основ процесса перехвата воздушных целей зенитными управляемыми ракетами. В работе рассматриваются основные методы наведения ЗУР, системы координат для описания движения, классификация систем управления, а также физические аспекты поражения цели боевой частью ракеты.