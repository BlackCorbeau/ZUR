\section{Заключение}
В ходе работы были систематизированы физические основы перехвата воздушных целей зенитными управляемыми ракетами.
\begin{enumerate}
    \item \textbf{Методы наведения} (пропорциональное сближение, параллельное сближение, трёхточечный и метод половинного спрямления) определяют траекторию ракеты и требуемую точность управления. Каждый метод имеет свои преимущества и ограничения, что обуславливает его применение в конкретных условиях перехвата.
    \item \textbf{Системы координат} (биконическая, связанная, стартовая, скоростная) обеспечивают математическое описание движения цели и ракеты, необходимое для расчёта управляющих команд.
    \item \textbf{Системы управления} (командные, теленаведение, самонаведение, комбинированные) реализуют выбранный метод наведения на практике. Современные ЗРК часто используют комбинированные системы для повышения точности и помехоустойчивости.
    \item \textbf{Физические основы поражения цели} включают конструкцию боевой части, тип взрывателя и учёт уязвимости цели. Эффективность поражения зависит от точности подрыва и согласованности работы всех элементов системы.
\end{enumerate}
Таким образом, успешный перехват воздушной цели является результатом сложного взаимодействия методов наведения, систем координат, управления и поражающих элементов. Развитие этих компонентов продолжает определять эволюцию зенитных ракетных комплексов.