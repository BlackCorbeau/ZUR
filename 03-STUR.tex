\section{Системы управления зенитных управляемых ракет}

\subsection{Командные системы телеуправления}
Принцип действия командных систем заключается в формировании управляющих команд на наземном (или корабельном) пункте управления (ПУ) на основе данных о цели и ракете с последующей их передачей на борт по радиоканалу. Выделяют две основные разновидности таких систем.

\subsubsection{Система телеуправления первого вида (ТУ-1)}
В системе ТУ-1 вся информация измеряется наземными средствами. Устройство сопровождения цели (УСЦ) измеряет координаты цели, а после пуска ракеты устройство сопровождения ракеты (УСР) измеряет её координаты. Эти данные поступают в устройство формирования команд (УФК), где вычисляется управляющее воздействие. Сформированные команды через командную радиолинию управления (КРУ) передаются на борт ракеты, где дешифруются и исполняются автопилотом.
\begin{figure}[H]
	\centering
	\includegraphics[width=\textwidth]{03-tu-1.png}
	\caption{Командная система телеуправления 1-го вида}
\end{figure}

\subsubsection{Система телеуправления второго вида (ТУ-2)}
Система ТУ-2  отличается тем, что информация о цели получается бортовым устройством ракеты — координатором или бортовым радиопеленгатором (БРП). После предварительной обработки данные о цели передаются по радиолинии на ПУ и вводятся в УФК. Координаты ракеты, как и в системе ТУ-1, измеряются наземным УСР. Дальнейший процесс формирования и передачи команд управления аналогичен.
\begin{figure}[H]
	\centering
	\includegraphics[width=\textwidth]{03-tu-2.png}
	\caption{Командная система телеуправления 2-го вида}
\end{figure}
\paragraph*{Примечание: Бинарное наведение}
В современных ЗРК системы ТУ-2 в чистом виде используются редко. Для повышения надёжности и точности применяется \emph{бинарное наведение}, при котором координаты цели одновременно измеряются как бортовыми средствами ракеты, так и наземными средствами ПУ, с выбором источника, обеспечивающего наилучшую точность.

\subsection{Системы теленаведения (наведение по лучу)}
В системах теленаведения управляющие команды формируются непосредственно на борту ракеты. Величина команды пропорциональна отклонению ракеты от равносигнального направления радиолокационного луча («луча наведения»), формируемого пунктом управления. Различают:
\begin{itemize}
	\item \textbf{Однолучевые системы:} Ракета наводится по тому же лучу, что и сопровождается цель (луч УСЦНР — устройства сопровождения цели и наведения ракеты).
	\item \textbf{Двухлучевые системы:} Для наведения ракеты формируется отдельный луч устройством наведения ракеты (УНР).
\end{itemize}

\begin{figure}[H]
	\centering
	\includegraphics[width=\textwidth]{03-tele.png}
	\caption{Схема системы теленаведения: а-однолучевая, б-двухлучевая}
\end{figure}

\subsection{Системы самонаведения}
В системах самонаведения управление полётом осуществляется полностью автономно бортовой аппаратурой ракеты. Пункт управления в процессе наведения участия не принимает. Классификация систем самонаведения проводится по виду энергии, используемой для получения информации о цели.

\subsubsection{Активное самонаведение}
Источник облучения цели (радиолокатор) установлен непосредственно на борту ракеты. Бортовой координатор принимает сигнал, отражённый от цели, и измеряет параметр рассогласования.

\subsubsection{Полуактивное самонаведение}
Цель облучается внешним источником — станцией подсвета цели (СПЦ), размещённой на ПУ. Бортовой координатор ракеты принимает отражённый от цели сигнал.

\subsubsection{Пассивное самонаведение}
Для измерения параметров движения цели используется энергия, излучаемая самой целью (например, тепловая, радиотепловая, световая).

\begin{figure}[H]
	\centering
	\includegraphics[width=\textwidth]{03-self.png}
	\caption{схема систем самонаведения: a-активная, б-полуактивная, с-пассивная, }
\end{figure}

\subsection{Комбинированные системы управления}
Для обеспечения высокой точности наведения на больших дальностях в комплексах средней и большой дальности применяются комбинированные системы. В них на различных участках траектории полёта ракеты последовательно используются разные принципы управления.

\subsubsection{Типовые комбинации систем}
\begin{itemize}
	\item Командное телеуправление (на начальном участке) + Самонаведение (на конечном участке).
	\item Теленаведение (по лучу на маршевом участке) + Самонаведение (на терминальном участке).
	\item Инерциальное наведение (на основном участке) + Телеуправление 2-го вида или Самонаведение (на конечном участке).
\end{itemize}
Использование комбинированных систем позволяет оптимально распределить функции между носителем и ракетой, повысив общую эффективность комплекса.